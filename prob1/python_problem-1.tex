\documentclass[10pt, letterpaper, acro-macros]{python-problem}

\usepackage{url}

\problemset{1}
\date{}

\begin{document}

\maketitle

\section*{Rules}

This exercise can be performed completely within \code{ipython} with out the
pylab extension. \emph{At no point can you store what Bill says! You cannot
edit Bill either; that would be cruel}. All interactions must be through
talking with him. The following are invalid.

\begin{pycode*}{gobble=2}
  a = talkToBill()
  talkToBill().some_function() # Bill's output is temporarily stored in a chain.
  talkToBill().some_property   # See above.
\end{pycode*}

All of the code problems (except the \textsc{rpn} calculator) can be solved
in one legible line of code (ignoring import statements). This means
\code{def} is not required, but also not an invalid solution.

Hints: \url{http://goo.gl/7JRUpU},
\url{http://goo.gl/frBLtZ}, \url{http://goo.gl/QGjZ7k}.


\section*{Problem}

Bill is a travelling salesman with a briefcase filled with goodies that he
is attempting to sell to you. However, Bill is a strange character. He only
seems to accept programming statements, and only one line expressions at
that. What can I say, \textsc{cs} people are kinda \emph{weird}.

Your only interface to Bill is the \code{talkToBill} function, and you
forgot a notepad so you can't write down what he says (see above).

When you send him no input (google for these answers),

\begin{enumerate}
  \item What type of data structure does he return? Write a new function
    that returns the same data (create a Bill simulator).
  \item What is the access cost for this type of data structure? Why?
  \item What is the insert cost for this type of data structure? Why?
  \item What are the properties that index must hold when
    inserting an element into this data structure? Why?
  \item What other scenarios would this data structure be useful for?
\end{enumerate}

You begin to wonder some profound mathematics questions to yourself. You
decide to ask Bill, keeping in mind that he only accepts one statement
inputs (an example is given in the help of \code{talkToBill}).

\begin{enumerate}[resume]
  \item ``What is Bill's default input?''
  \item ``What items can I get from Bill?'' (in sorted order)
  \item ``If I chose one of each item from Bill, how much would that
    cost?''
  \item ``Can I get Bill to return a list of tuples, where the tuple
    contains the item and the cost?''
\end{enumerate}

Bill is so predictable. In fact, this makes him very vulnerable. He can do
all sorts of things, if you just ask in the right way.

\begin{enumerate}[resume]
  \item How can you get him to write out your name?
  \item Make him tell you ``NAME is pretty'' for any input name.
  \item How can you get him to print out the date?
\end{enumerate}

You want to ask him more complicated questions but sadly forgot your pad at
the office. But, being quite the clever programmer, you decide there is an
alternative way to get him to remember things for you. This opens a world of
questions that couldn't be asked before (or not easily).

\begin{enumerate}[resume]
  \item For all the items Bill blabs on about, reverse the name of the item.
    Additionally, remove all items that are cheaper than \$2. The resulting
    data type should be the same as when Bill is called with no input.
  \item Make Bill output both the price and the current date he sold you the
    item.
  \item Add five hilarious items to his list.
  \item For each item, choose a random number between 0 and 10 of that item.
    How much would that cost? (Additionally, how could you do this in one
    statement like in the second section?)
\end{enumerate}

Finally, out of boredom, you decide to turn Bill into a calculator! He
supports the following operations: \code{+, -, *, /}. You set him up to work
like a Reverse Polish Notation (\textsc{rpn}) calculator, where he accept a
list of numbers and operators and returns a result.


\section*{Appendix}

Below is the function declaration/help for the \code{talkToBill} function.

\begin{pycode*}{gobble=2}
  def talkToBill(f):
      ''' Bill, the weird salesman

      Bill only responds to single statement inputs.

      Parameters
      ----------
      f : statement to ask.
          What do you want to ask Bill?

      Returns
      -------
      Depends!
      '''
\end{pycode*}


%%%%%%%%%%%%%%%%%%%%%%%%%%%%%%%%%%%%%%%%%%%%%%%%%%%%%%%%%%%%%%%%%%%%%%%
%                              Solutions                              %
%%%%%%%%%%%%%%%%%%%%%%%%%%%%%%%%%%%%%%%%%%%%%%%%%%%%%%%%%%%%%%%%%%%%%%%
\beginsolution

\section*{Solutions}

\begin{enumerate}
  \item A dictionary. A function that accomplishes the same effect as
    \code{talkToBill} would be as follows.

    \begin{pycode*}{gobble=6}
      def billSimulator():
          return {'cabbage': 2.5, 'aubergine': 3.0,
                  'pear': 1.5, 'apple': 1.25}
    \end{pycode*}


  \item The access cost is \BigO{1}. This is because the dictionary is really
    a hash table. See \url{https://en.wikipedia.org/wiki/Hash_table}.

    Briefly, what this means is that there is a hash function which takes an
    input and returns some output that is within some modulus. This returned
    value is used as a lookup into an array. If the hash function has
    constant time to compute, then the lookup is also in constant time since
    we can access any element in an array in constant time.

    The access of a dictionary can be emulated (in pseudocode) as follows,
    given that the dictionary acted like a list.

    \begin{pycode*}{gobble=6}
      def lookup(dictionary, key):
          return dictionary[hash(key)]
    \end{pycode*}


  \item The insertion cost is \BigO{1}. Similar to the reasoning above,
    since we can find the insertion point using the hash function in
    constant time, we can insert an item into that location in constant
    time. There is a bit of a hick-up related to if two different keys map
    to the same lookup index value, which is discussed more in detail in the
    \textsc{url} above.

    The insertion of a dictionary can be emulated (in pseudocode) as follows,
    given that the dictionary acted like a list.

    \begin{pycode*}{gobble=6}
      def insert(dictionary, key, item):
          dictionary[hash(key)].append(item)
    \end{pycode*}


  \item The index must be \emph{immutable}. This means that once the value is
    created in memory, it cannot be altered, only read. An example of this
    type of variable is strings, integers, floats, and tuples (why might
    this property of tuples be amazingly useful?). In addition, the index
    type must implement the \code{hash} function, which allows the code
    above to work correctly.


  \item Dictionaries are insanely useful in Python. Classes in Python are, at
    their core, just dictionaries. You can try this out by seeing what dot
    operations are available on a class. This also means that classes can
    be mutated after they are created, and that classes can be created
    programmatically (a technique known as meta-programming).

    Additionally, dictionaries are great ways to store any data as key value
    pairs. This setup is often used in high performance databases, or for
    creating simple ways to pass around related parameters in a program.

    Additionally, dictionaries are great ways to store structures like
    \textsc{xml} and \textsc{json}.

    More information on dictionaries can be found here:
    \url{http://openbookproject.net/thinkcs/python/english3e/dictionaries.html}


  \item By default, the message sent to Bill is the identity function (known
    as \code{id} in other languages). It is equivalent to the following
    line.

    \begin{pycode*}{gobble=6}
      talkToBill(lambda x: x)
    \end{pycode*}


  \item Simply pass the keys of Bill's dictionary to the \code{sorted}
    function. \url{https://docs.python.org/3.3/howto/sorting.html}

    \begin{pycode*}{gobble=6}
      talkToBill(lambda x: sorted(x.keys()))
    \end{pycode*}


  \item The result of buying one of each item is \$8.25. Note that the type
    that \code{x.values} returns is an iterator, which returns the elements
    of the dictionary one by one. A good reference can be found here:\\
    \url{http://answers.oreilly.com/topic/1576-new-iterables-in-python-3-0/}

    \begin{pycode*}{gobble=6}
      talkToBill(lambda x: sum(x.values()))
    \end{pycode*}


  \item This is really a test in list comprehensions. The following lines are
    equivalent.

    \begin{pycode*}{gobble=6}
      talkToBill(lambda x: [(k, v) for k, v in x.items()])
      talkToBill(lambda x: [x for x in x.items()])
      talkToBill(lambda x: list(x.items()))
    \end{pycode*}


  \item Just ignore the input and replace it with what you want. Note that I
    use the underscore to represent an unused argument. This is a somewhat
    common programming practice.

    \begin{pycode*}{gobble=6}
      talkToBill(lambda _: "ryan")
    \end{pycode*}


  \item Printing an arbitrary name requires a small trick. I said you can't
    make variables. \emph{Functions} are perfectly OK (we are ignoring
    that functions are variables at the moment). This trick can be used in
    problems 12-15.

    \begin{pycode*}{gobble=6}
      def sayPretty(name): return "{} is pretty".format(name)
      talkToBill(lambda _: sayPretty("Sussy"))
    \end{pycode*}

    An different, perhaps more correct solution (since no code is required
    outside of the \code{talkToBill} call) is to create an anonymous
    function and call it directly.

    \begin{pycode*}{gobble=6}
      talkToBill(lambda _: (lambda _, name: "{} is pretty".format(name))(_, "Sussy"))
    \end{pycode*}

    Note that the \code{sayPretty} function could also be defined by using
    the \code{\%} operator. However, this is a deprecated feature, and
    all string formatting should be done with the new \code{str.format}
    function. See the following \textsc{url} for more details.
    \url{https://docs.python.org/3.1/library/string.html}


  \item The datetime module is a bit screwy, but this is how you could code
    this. The result of \code{datetime.now} is a datetime object.

    \begin{pycode*}{gobble=6}
      from datetime import datetime
      talkToBill(lambda _: datetime.now())
    \end{pycode*}


  \item There are two methods to both filter out and reverse the key
    strings. The first is to create a new function that will hold
    intermediaries of two different dictionary comprehensions.

    \begin{pycode*}{gobble=6}
      def reverse_and_filter(x):
          filtered = {k: v for k, v in x.items() if v > 2}
          return {k[::-1]: v for k, v in filtered.items()}
      talkToBill(lambda x: reverse_and_filter(x))
    \end{pycode*}

    The second is a slightly advanced dictionary comprehension.
    \begin{pycode*}{gobble=6}
      talkToBill(lambda x:
          { k[::-1]: v for k, v in x.items() if v > 2 } )
    \end{pycode*}


  \item Again, the values can be modified to include a tuple by using a
    dictionary comprehension.

    \begin{pycode*}{gobble=6}
      from datetime import datetime
      talkToBill(lambda x:
          { k: (v, datetime.now()) for k, v in x.items() } )
    \end{pycode*}


  \item This solution leverages the update function in a dictionary.

    \begin{pycode*}{gobble=6}
      def addToDict(x):
          x.update({'a': 1, 'b': 2, 'c': 3, 'd': 4, 'e': 5})
          return x
      talkToBill(lambda x: addToDict(x))
    \end{pycode*}

    Or if you are sadistic:

    \begin{pycode*}{gobble=6}
      from itertools import chain
      talkToBill(lambda x:
        dict(chain(x.items(), {'a': 1, 'b': 2, 'c': 3, 'd': 4, 'e': 5}.items())))
    \end{pycode*}


  \item This one can be done in either a more traditional sense.

    \begin{pycode*}{gobble=6}
      import random

      def sumItems(x):
          cost = 0

          for v in x.values():
              cost += random.randint(0, 10) * v

          return cost

      talkToBill(lambda x: sumItems(x))
    \end{pycode*}

    Or with a generator.
    \begin{pycode*}{gobble=6}
      import random
      talkToBill(lambda x:
          sum(v*random.randint(0, 10) for v in x.values()))
    \end{pycode*}


  \item To create the \textsc{rpn} calculator, all you need is a stack and
    the input list (see
    \url{https://en.wikipedia.org/wiki/Reverse_Polish_notation}). The code
    is shown below.

    \pyfile{rpn.py}

\end{enumerate}

\end{document}
